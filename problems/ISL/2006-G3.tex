\documentclass[14pt]{article}

% Packages
\usepackage{ezra}

% Title
\title{2006 G3}
\author{Ezra Guerrero Alvarez}

% Formatting
\setlength{\parindent}{0pt}

%document
\begin{document}
\maketitle
	
\section*{2006 G3}

\begin{statement}{2006 G3}
	Let $ ABCDE$ be a convex pentagon such that
	\[
	\angle BAC = \angle CAD = \angle DAE\qquad \text{and}\qquad \angle ABC = \angle ACD = \angle ADE.
	\]
	The diagonals $BD$ and $CE$ meet at $P$.  Prove that the line $AP$ bisects the side $CD$.
\end{statement}
Note that there is a spiral similarity with center $A$ taking $B\rightarrow C\rightarrow D\rightarrow E$. Thus, the spiral similarity takes $\seg{BD}$ to $\seg{CE}$. Therefore, since $P=\seg{BD}\cap\seg{CE}$, we have $AEDP$ and $ABCP$ cyclic. Thus, since $\angle CBP = \angle DCP$ from the spiral similarity, we have
\[ \angle DCP = \angle CBP = \angle CAP, \]
giving $(ABCP)$ tangent to $\seg{CD}$. Analogously, $(AEDP)$ is tangent to $\seg{CD}$. Thus, since $\seg{AP}$ is the radical axis of these two circles, it bisects their common external tangent $\seg{CD}$. $\blacksquare$
	
\end{document}
