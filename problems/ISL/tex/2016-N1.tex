\documentclass[14pt]{article}

% Packages
\usepackage{ezra}

% Title
\title{2016 N1}
\author{Ezra Guerrero Alvarez}

% Formatting
\setlength{\parindent}{0pt}

%document
\begin{document}
\maketitle
	
\section*{2016 N1}

\begin{statement}{2016 N1}
	For any positive integer $k$, denote the sum of digits of $k$ in its decimal representation by $S(k)$. Find all polynomials $P(x)$ with integer coefficients such that for any positive integer $n \geq 2016$, the integer $P(n)$ is positive and
	\[ S(P(n)) = P(S(n)). \]
\end{statement}
We claim the only solutions are $P(x)=1,2,\ldots,9$ and $P(x)=x$, which clearly work. We show they are the only ones.\\
Let $P(x)=\sum a_ix^i$. Setting $x=10^k$ for $k\ge4$,
\[ \sum a_i=P(1)=S(P(10^k))\le\sum S(a_i). \]
This is clearly impossible unless the $a_i$ are digits. Now, setting $x=9\cdot 10^k$ for $k\ge 3$, 
\[ \sum 9^ia_i=P(9)=S(P(9\cdot10^k))\le\sum S(9^ia_i). \]
This is clearly impossible unless $a_i=0$ for all $i\ge2$ and $a_1\in\{0,1\}$. If $a_1=0$ it follows $a_0\in\{1,2,\ldots,9\}$. If $a_1=1$, then
\[ S(n+a_0)=S(n)+a_0. \]
Taking $n=2019$, it's easy to see $a_0=0$. This concludes the proof. $\blacksquare$
	
\end{document}
