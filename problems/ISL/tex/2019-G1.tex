\documentclass[14pt]{article}

% Packages
\usepackage{ezra}

% Title
\title{2019 G1}
\author{Ezra Guerrero Alvarez}

% Formatting
\setlength{\parindent}{0pt}

%document
\begin{document}
\maketitle
	
\section*{2019 G1}

\begin{statement}{2019 G1}
	Let $ABC$ be a triangle.
	A circle passes through $A$,
	intersects $\seg{AB}$ and $\seg{AC}$ again at
	$D$ and $E$ respectively,
	and intersects $\seg{BC}$ at $F$ and $G$,
	with $BF < BG$.
	Let $T$ be a point such that $\seg{FT}$
	is tangent to the circumcircle of $\triangle BDF$
	and $\seg{GT}$ is tangent to the circumcircle
	of $\triangle CEG$.
	Prove that $\seg{AT} \parallel \seg{BC}$.
\end{statement}
We use directed angles$\mod 180\dg$. Using the tangency and cyclic pentagon $ADFGE$, we have
\begin{align*}
	\dang GFT = \dang BFT = \dang BDF = \dang ADF = \dang AGF \\
	\dang TGF = \dang TGC = \dang GEC = \dang GEA = \dang GFA 
\end{align*}
Adding these, we obtain $-\dang FTG = -\dang FAG$, so $T$ lies on the circumcircle of $\triangle AFG$. However, we also have $\dang GFT = \dang AGF$, so $ATGF$ must be an isosceles trapezoid with $\seg{AT}\parallel\seg{GF}$, which is what we wanted to prove. $\blacksquare$
	
\end{document}
