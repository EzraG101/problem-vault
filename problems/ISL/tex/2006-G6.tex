\documentclass[14pt]{article}

% Packages
\usepackage{ezra}

% Title
\title{2006 G6}
\author{Ezra Guerrero Alvarez}

% Formatting
\setlength{\parindent}{0pt}

%document
\begin{document}
\maketitle
	
\section*{2006 G6}

\begin{statement}{2006 G6}
	Circles $\omega_1$ and $\omega_2$ with centers $O_1$ and 
	$O_2$ are externally tangent at point $D$ and internally 
	tangent to a circle $\omega$ at points $E$ and $F$ 
	respectively. Line $t$ is the common tangent of $\omega_1$ 
	and $\omega_2$ at $D$. Let $\seg{AB}$ be the diameter of 
	$\omega$ perpendicular to $t$, so that $A, E, O_1$ are on 
	the same side of $t$. Prove that lines $AO_1$, $BO_2$, $EF$ 
	and $t$ are concurrent.
\end{statement}
Let $O$ be the center of $\omega$ and let $P$ be the 
intersection of the $O-$median of $\triangle OO_1O_2$ and 
$\seg{EF}$. We will show $P$ lies on the $4$ lines, which 
implies they are concurrent. First, note that since 
$O_1D=O_1E,O_2D=O_1F$, and $OE=OF$, circle $(DEF)$ is tangent to 
all three sides of $\triangle OO_1O_2$. Since $D$ lies between 
$O_1$ and $O_2$ and $E$ and $F$ lie outside of $\seg{OO_1}$ and 
$\seg{OO_2}$ respectively, it follows $(DEF)$ is the 
$O-$excircle of $\triangle OO_1O_2$. \\
We now proceed with barycentric coordinates. \bary O{O_1}{O_2} 
It follows that $D=(0:s-b:s-c),E=(-(s-c):s:0),F=(-(s-b):0:s)$, 
where $s=\frac{a+b+c}2$. Hence, line $EF$ has equation
\[ sx+(s-c)y+(s-b)z=0. \]
Since $P$ lies on this line and on the $O-$median, it has 
coordinates $(-a:s:s)$. Now, note that since $t$ is tangent to 
both $\omega_1$ and $\omega_2$, we have $t\perp\seg{O_1O_2}$. 
Since $t\perp\seg{AB}$ by construction, it follows 
$\seg{AB}\parallel\seg{O_1O_2}$. Also, since $O\in\seg{AB}$, it 
follows $A$ and $B$ have coordinates of the form $(1,t,-t)$. 
Now, note that $\omega$ has radius $OE=s$. Hence, $OA^2=s^2$. By 
the barycentric distance formula, since $\ray{OA}=(0,t,-t)$, we 
see $s^2=a^2t^2$. Hence, $t=\pm\frac sa$. Indeed, $B$ satisfies 
the same equation, so one solution corresponds to $A$ and the 
other to $B$. Since $A$ lies on the same side of $t$ as $O_1$, 
we have $A=\left(1,\frac sa,-\frac sa\right)=(a:s:-s)$. 
Similarly, $B=(a:-s:s)$. Then, points on the cevian $O_1A$ are 
given by $(a:k:-s)$ and points on the cevian $O_2B$ are given by 
$(a:-s:k)$. It is easy to see $P=(a:-s:-s)$ is of these forms, 
so $P$ lies on both lines. Finally, we must see $P\in t$. This 
is equivalent to showing $\seg{PD}\perp\seg{O_1O_2}$. Since 
$\ray{O_2O_1}=(0,1,-1)$ and $\ray{PD}=\left(\frac 
a{b+c},\frac{s-b}a-\frac s{b+c},\frac{s-c}a-\frac 
s{b+c}\right)$, from Evan's Favorite Forgotten Trick, it 
suffices to show
\[ a^2\left(\frac{s-c}a-\frac s{b+c}+\frac 
s{b+c}-\frac{s-b}a\right)+b^2\left(-\frac 
a{b+c}\right)+c^2\left(\frac a{b+c}\right)=0.\]
Indeed, simplifying the left hand side we obtain
\[ a(b-c)+a\cdot\frac{c^2-b^2}{b+c}=0 \]
as desired. Thus, $P$ lies on $AO_1,BO_2,EF$, and $t$, so these 
$4$ lines are concurrent. $\blacksquare$ 
	
\end{document}
