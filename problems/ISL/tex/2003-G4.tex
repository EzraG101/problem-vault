\documentclass[14pt]{article}

% Packages
\usepackage{ezra}

% Title
\title{2003 G4}
\author{Ezra Guerrero Alvarez}

% Formatting
\setlength{\parindent}{0pt}

%document
\begin{document}
\maketitle
	
\section*{2003 G4}

\begin{statement}{2003 G4}
	Let $\Gamma_1$, $\Gamma_2$, $\Gamma_3$, $\Gamma_4$ be distinct circles such that $\Gamma_1$, $\Gamma_3$ are externally tangent at $P$, and $\Gamma_2$, $\Gamma_4$ are externally tangent at the same point $P$. Suppose that $\Gamma_1$ and $\Gamma_2$; $\Gamma_2$ and $\Gamma_3$; $\Gamma_3$ and $\Gamma_4$; $\Gamma_4$ and $\Gamma_1$ meet at $A$, $B$, $C$, $D$, respectively, and that all these points are different from $P$. Prove that
	\[ \frac{AB\cdot BC}{AD\cdot DC}=\frac{PB^2}{PD^2}. \]
\end{statement}
Let $\gamma_1,\gamma_2,\gamma_3,\gamma_4,A',B',C',D'$ be the images of $\Gamma_1,\Gamma_2,\Gamma_3,\Gamma_4,A,B,C,D$ under an inversion with center $P$ and radius $1$. From the given conditions, we see the quadrilateral formed by lines $\gamma_1,\gamma_2,\gamma_3,\gamma_4$ is a parallelogram. Furthermore, this quadrilateral is $A'B'C'D'$. Therefore, $A'B'=D'C'$. From the inversion distance formula, this gives
\[ \frac{AB}{PA\cdot PB}=\frac{DC}{PD\cdot PC}, \]
or $\dfrac{AB}{DC}=\dfrac{PA}{PC}\cdot\dfrac{PB}{PD}$. Analogously, we obtain
\[ \frac{BC}{AD}=\frac{PC}{PA}\cdot\frac{PB}{PD}. \]
Multiplying these two equations,
\[ \frac{AB\cdot BC}{AD\cdot DC}=\frac{PA}{PC}\cdot\frac{PB}{PD}\cdot\frac{PC}{PA}\cdot\frac{PB}{PD}=\frac{PB^2}{PD^2}, \]
as desired. $\blacksquare$
	
\end{document}
