\documentclass[14pt]{article}

% Packages
\usepackage{ezra}

% Title
\title{2003 G2}
\author{Ezra Guerrero Alvarez}

% Formatting
\setlength{\parindent}{0pt}

%document
\begin{document}
\maketitle
	
\section*{2003 G2}

\begin{statement}{2003 G2}
	Three distinct points $A$, $B$, and $C$ are fixed on a line in this order.  Let $\Gamma$ be a circle passing through $A$ and $C$ whose center does not lie on the line $AC$.  Denote by $P$ the intersection of the tangents to $\Gamma$ at $A$ and $C$.  Suppose $\Gamma$ meets the segment $PB$ at $Q$.  Prove that the intersection of the bisector of $\angle AQC$ and the line $AC$ does not depend on the choice of $\Gamma$.
\end{statement}
By properties of harmonic quads we find $\seg{BQ}$ is the $Q-$symmedian of $\triangle AQC$. Therefore,
\[ \frac{AB}{BC}=\left(\frac{AQ}{QC}\right)^2. \]
Thus, if the bisector of $\angle AQC$ meets $\seg{AC}$ at $T$, then
\[ \frac{AT}{TC}=\frac{AQ}{QC}=\sqrt{\frac{AB}{BC}}, \]
which does not depend on the choice of $\Gamma$, as we wanted to show. $\blacksquare$ 
	
\end{document}
