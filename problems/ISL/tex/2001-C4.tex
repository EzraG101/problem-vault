\documentclass[14pt]{article}

% Packages
\usepackage{ezra}

% Title
\title{2001 C4}
\author{Ezra Guerrero Alvarez}

% Formatting
\setlength{\parindent}{0pt}

%document
\begin{document}
\maketitle
	
\section*{2001 C4}

\begin{statement}{2001 C4}
	A set of three nonnegative integers $\{x,y,z\}$ with $x < y < z$ is called historic if $\{z-y,y-x\} = \{1776,2001\}$. Show that the set of all nonnegative integers can be written as the union of pairwise disjoint historic sets.
\end{statement}
We use the following greedy algorithm: In each step we take the least nonnegative integer $x$ that has not been selected and add the set $\{x,x+1776,x+1776+2001\}$ if it's available and $\{x,x+2001,x+1776,x+1776+2001\}$ otherwise. Also, when choosing a historic set, we color the least number \textcolor{red}{red}, the greatest \textcolor{blue}{blue} and the middle one \textcolor{green}{green}. It suffices to show that if in a step we pick $x$, then $x+1776+2001$ and one of $x+1776,x+2001$ have not been colored. Clearly they are not colored \textcolor{red}{red}, since they are greater than $x$. If $x+1776+2001$ is \textcolor{green}{green}, then one of $x+1776,x+2001$ must be \textcolor{red}{red}. If it is \textcolor{blue}{blue}, then $x$ would have already been chosen. In conclusion, $x+1776+2001$ is not colored. Now, if $x+2001$ were \textcolor{green}{green}, $x+225$ would be \textcolor{red}{red}, but this is not possible. Hence, if $x+2001$ is colored, it is \textcolor{blue}{blue}. Then, $x-1776$ is \textcolor{red}{red} and $x$ is \textcolor{green}{verde}, but $x$ was not colored. Therefore, we can always choose $x+2001$. Thus the algorithm never fails and $\NN$ is partitioned. $\blacksquare$
	
\end{document}
