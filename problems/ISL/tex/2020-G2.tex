\documentclass[14pt]{article}

% Packages
\usepackage{ezra}

% Title
\title{2020 G2}
\author{Ezra Guerrero Alvarez}

% Formatting
\setlength{\parindent}{0pt}

%document
\begin{document}
\maketitle
	
\section*{2020 G2}

\begin{statement}{2020 G2}
	Consider the convex quadrilateral $ABCD$.
	The point $P$ is in the interior of $ABCD$.
	The following ratio equalities hold:
	\[\angle PAD:\angle PBA:\angle DPA
	= 1:2:3
	= \angle CBP:\angle BAP:\angle BPC.\]
	Prove that the following three lines meet in a point:
	the internal bisectors of angles $\angle ADP$ and $\angle PCB$
	and the perpendicular bisector of segment $AB$.
\end{statement}
Let $R$ and $S$ be points on $\seg{DA}$ and $\seg{BC}$ such that $\olra{BR}$ and $\olra{AS}$ are the interior angle bisectors of $\angle PBA$ and $\angle BAP$ respectively. Then, the angle ratios give
\[ \angle PAR = \angle PBR,  \]
so $RABP$ is cyclic. Since $\angle RBA = \angle PBR$, we have $\angle RPA = \angle PAR$. Thus, $\angle PRD = 2\angle PAR$ from the exterior angle theorem. Also, from the angle ratios we have $\angle RPD = 2\angle PAR$. Hence, $\triangle DRP$ is isosceles with $D$ as its vertex, so the internal angle bisector of $\angle ADP$ is the perpendicular bisector of $\seg{RP}$. Analogously, $SBAP$ is cyclic and the interior angle bisector of $\angle PCB$ is the perpendicular bisector of $\seg{PS}$. Since $ARPSB$ is cyclic, the perpendicular bisectors of $\seg{RP},\seg{PS}$, and $\seg{AB}$ concur at its circumcenter, giving the desired result. $\blacksquare$ \\\\
\textbf{Remark.} I solved this in contest!!!
	
\end{document}
