\documentclass[14pt]{article}

% Packages
\usepackage{ezra}

% Title
\title{2011 G2}
\author{Ezra Guerrero Alvarez}

% Formatting
\setlength{\parindent}{0pt}

%document
\begin{document}
\maketitle
	
\section*{2011 G2}

\begin{statement}{2011 G2}
	Let $A_1A_2A_3A_4$ be a non-cyclic quadrilateral. Let $O_1$ and $r_1$ be the circumcentre and the circumradius of the triangle $A_2A_3A_4$. Define $O_2,O_3,O_4$ and $r_2,r_3,r_4$ in a similar way. Prove that
	\[
	\frac{1}{O_1A_1^2-r_1^2}+\frac{1}{O_2A_2^2-r_2^2}+\frac{1}{O_3A_3^2-r_3^2}+\frac{1}{O_4A_4^2-r_4^2}=0.
	\]
\end{statement}
\bary {A_1}{A_2}{A_3} Let $A_4=(p,q,r)$. Let $T=a^2qr+b^2rp+c^2pq$. Then, we find circle $A_1A_2A_4$ has equation
\[ -a^2yz-b^2zx-c^2xy+(x+y+z)\left(\frac Tr\cdot z\right)=0. \]
Therefore, $O_3A_3^2-r_3^2$, which is the power of $A_3$ with respect to this circle, equals
\[ \frac Tr. \]
Analogously,
\[ O_2A_2^2-r_2^2 = \frac Tq\,\text{and}\, O_1A_1^2-r_1^2 = \frac Tp. \]
Since $O_4A_4^2-r_4^2=-T$, we find the desired sum equals
\[ \frac pT + \frac qT + \frac rT - \frac 1T = \frac 1T - \frac 1T = 0, \]
as desired. $\blacksquare$
	
\end{document}
