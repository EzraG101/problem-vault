\documentclass[14pt]{article}

% Packages
\usepackage{ezra}

% Title
\title{1997 SL4}
\author{Ezra Guerrero Alvarez}

% Formatting
\setlength{\parindent}{0pt}

%document
\begin{document}
\maketitle
	
\section*{1997 SL4}

\begin{statement}{1997 SL4}
	An $ n \times n$ matrix whose entries come from the set $ S = \{1, 2, \ldots , 2n - 1\}$ is called a \textit{silver matrix} if, for each $ i = 1, 2, \ldots , n$, the $ i$-th row and the $ i$-th column together contain all elements of $ S$. Show that:
	\begin{enumerate}[label=(\alph*)]
		\item there is no silver matrix for $ n = 1997$;
		\item silver matrices exist for infinitely many values of $ n$.
	\end{enumerate}
\end{statement}
Suppose we have an $n\times n$ ($n>1$) silver matrix $A$. Weight each element $a_{ij}$ of the matrix with a $w_{ij}=1$ if $i=j$ and $w_{ij}=2$ if $i\ne j$. From the definition of a silver matrix, 
\[ C:=\sum_{a_ij=r}w_{ij} \]
is the same for all $r\in S=\{1,2,\ldots,2n-1\}$. Note that $2n-1>n$, so that there is some element of $S$ that does not appear in the main diagonal. From this element, it follows $C$ is even, since $w_{ij}$ will always be $2$. Therefore, it follows that if any element of $S$ must show up an even number of times in the main diagonal. This is clearly impossible if $n$ is odd, which proves (a).\\
Now, we provide an inductive proof that silver matrices always exist for $n$ a power of $2$, which will prove (b). We have the following construction for $n=2$:
\[ \begin{bmatrix}
	A&C\\
	B&A
\end{bmatrix}, \]
where $\{A,B,C\}=\{1,2,3\}$. The crucial bit is that the main diagonal consists of the same element. For our inductive hypothesis, assume a silver matrix exists for $n=2^k$, with the main diagonal being the same element. Now, we construct a silver matrix for $n=2^{k+1}$ with the main diagonal being the same element, which will conclude the induction. Let $M_1$ be a $2^k\times 2^k$ silver matrix, but replace $S$ with
\[S_1:=\{1,2,\ldots,2^k-1,2^{k+2}-2^k,\ldots,2^{k+2}-1\}.\]
Similarly, let $M_2$ and $M_3$ be $2^k\times 2^k$ silver matrices, but replacing $S$ with
\[ S_2:=\{2^k,\ldots,2^{k+2}-2^k-2\},S_3:=\{2^k+1,\ldots,2^{k+2}-2^k-1\} \]
respectively. $M_2$ and $M_3$ will be identical, except for their main diagonal. While $M_2$ has its main diagonal all equal to $2^k$, $M_3$ has its main diagonal all equal to $2^{k+2}-2^k-1$. We claim
\[ M := \begin{bmatrix}
	M_1&M_2\\
	M_3&M_1
\end{bmatrix} \]
is a silver matrix. Indeed, looking at the $i$th row and column, all elements of $S_1$ appear exactly once because of $M_1$, all elements of $S_2\cap S_3$ appear exactly once since $M_2$ and $M_3$ are identical except for the main diagonal, and $2^k,2^{k+2}-2^k-1$ appear exactly once because of the main diagonals of $M_2$ and $M_3$. Hence, $M$ is a silver matrix as desired. $\blacksquare$
	
\end{document}
