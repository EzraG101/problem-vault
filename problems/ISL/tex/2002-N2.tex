\documentclass[14pt]{article}

% Packages
\usepackage{ezra}

% Title
\title{2002 N2}
\author{Ezra Guerrero Alvarez}

% Formatting
\setlength{\parindent}{0pt}

%document
\begin{document}
\maketitle
	
\section*{2002 N2}

\begin{statement}{2002 N2}
	Let $n\geq2$ be a positive integer, with divisors $1=d_1<d_2<\,\ldots<d_k=n$.  Prove that $d_1d_2+d_2d_3+\,\ldots\,+d_{k-1}d_k$ is always less than $n^2$, and determine when it is a divisor of $n^2$.
\end{statement}
Note that
\[ \sum d_id_{i+1}=n^2\sum\frac1{d_id_{i+1}}\le n^2\sum\frac{d_{i+1}-d_i}{d_id_{i+1}}=n^2\sum\left(\frac1{d_i}-\frac1{d_{i+1}}\right)=n^2\left(1-\frac1n\right)<n^2. \]
Now, we show this sum only divides $n^2$ when $n$ is prime. Indeed, when $n$ is prime the sum equals $n$, so it divides $n^2$ trivially. Otherwise, if $n$ is composite, the sum is strictly greater than $d_{k-1}d_k=n^2/d_2$. However, since $d_2$ is the smallest prime factor of $n$, and consequently of $n^2$, there cannot be a divisor of $n^2$ between $n^2/d_2$ and $n^2$ exclusive. Hence, the sum does not divide $n^2$ when $n$ is composite. $\blacksquare$
	
\end{document}
