\documentclass[14pt]{article}

% Packages
\usepackage{ezra}

% Title
\title{2004 G2}
\author{Ezra Guerrero Alvarez}

% Formatting
\setlength{\parindent}{0pt}

%document
\begin{document}
\maketitle
	
\section*{2004 G2}

\begin{statement}{2004 G2}
	Let $\Gamma$ be a circle and let $d$ be a line such that $\Gamma$ and $d$ have no common points. Further, let $AB$ be a diameter of the circle $\Gamma$; assume that this diameter $AB$ is perpendicular to the line $d$, and the point $B$ is nearer to the line $d$ than the point $A$. Let $C$ be an arbitrary point on the circle $\Gamma$, different from the points $A$ and $B$. Let $D$ be the point of intersection of the lines $AC$ and $d$. One of the two tangents from the point $D$ to the circle $\Gamma$ touches this circle $\Gamma$ at a point $E$; hereby, we assume that the points $B$ and $E$ lie in the same halfplane with respect to the line $AC$. Denote by $F$ the point of intersection of the lines $BE$ and $d$. Let the line $AF$ intersect the circle $\Gamma$ at a point $G$, different from $A$. \\\\
	Prove that the reflection of the point $G$ in the line $AB$ lies on the line $CF$.
\end{statement}
We use directed angles mod $180\dg$. Let $M$ and $E'$ be the intersections of $\seg{AM}$ and $\seg{AE}$ with $d$, respectively. Note that $AEMF$ is cyclic with diameter $\seg{AF}$. We have
\[ \dang DEF = \dang DEB = \dang EAB = \dang EAM = \dang EFM = \dang EFD, \]
so $DE=DF$. Since $\triangle FEE'$ is right, it follows $D$ is the circumcenter of $(FEE')$. Now, note that the inversion with center $A$ and radius $\sqrt{AC\cdot AD}$ sends $\Gamma$ to $d$, implying $AG\cdot AF = AE\cdot AE'$ so $G\in(FEE')$. Therefore, $DG=DE$ and $\seg{DG}$ is tangent to $\Gamma$. By tangents,
\[ -1=(A,C;E,G)\stackrel F=(G,\seg{FC}\cap\Gamma;B,A). \]
Since $\seg{AB}$ is a diameter, it follows $\seg{FC}$ passes through the reflection of $G$ in line $\seg{AB}$, as desired. $\blacksquare$
	
\end{document}
