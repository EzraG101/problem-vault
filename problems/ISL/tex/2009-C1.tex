\documentclass[14pt]{article}

% Packages
\usepackage{ezra}

% Title
\title{2009 C1}
\author{Ezra Guerrero Alvarez}

% Formatting
\setlength{\parindent}{0pt}

%document
\begin{document}
\maketitle
	
\section*{2009 C1}

\begin{statement}{2009 C1}
	Consider $2009$ cards, each having one gold side and one black side,
	lying on parallel on a long table.
	Initially all cards show their gold sides.
	Two players, standing by the same long side of the table,
	play a game with alternating moves.
	Each move consists of choosing a block of $50$ consecutive cards,
	the leftmost of which is showing gold, and
	turning them all over,
	so those which showed gold now show black and vice versa.
	The last player who can make a legal move wins.
	
	\begin{enumerate}[label=(\alph*)]
		\item Does the game necessarily end?
		\item Does there exist a winning strategy for the starting player?
	\end{enumerate}
\end{statement}
We claim the game must end and that the starting player loses regardless of how the game is being played. Label the cards $1,2,\ldots 2009$ sequentially from left to right. Then, assign card $i$ a $1$ if it is gold and a $0$ otherwise. Concatenating the resulting number, we obtain a binary string, which is easily seen to be strictly decreasing after each move. Since it is always nonnegative, the game must eventually end. Now, consider cards $10, 60, \ldots, 1960$. For the game to have ended, these $40$ cards must be black. Also, note that every move affects exactly one of them. Thus, the number of black cards among these changes parity each move. Since there must be $40$ black cards when the game ends and we start with $0$, an even number of moves must have transpired, so the starting player loses. $\blacksquare$
	
\end{document}
