\documentclass[14pt]{article}

% Packages
\usepackage{ezra}

% Title
\title{2017 G3}
\author{Ezra Guerrero Alvarez}

% Formatting
\setlength{\parindent}{0pt}

%document
\begin{document}
\maketitle
	
\section*{2017 G3}

\begin{statement}{2017 G3}
	Let $O$ be the circumcenter of an acute triangle $ABC$. Line $OA$ intersects the altitudes of $ABC$ through $B$ and $C$ at $P$ and $Q$, respectively. The altitudes meet at $H$. Prove that the circumcenter of triangle $PQH$ lies on a median of triangle $ABC$.
\end{statement}
Let $T$ be the circumcenter of $\triangle PQH$. Also, denote by $E,F,M$ the feet of the altitudes from $B$ and $C$ and the midpoint of $\seg{BC}$, respectively. First, 
\[ \angle PQH = \angle QAC + \angle ACQ = 90\dg-\angle B+90\dg-\angle A=\angle C. \]
Therefore, $\angle THE = \angle THP = 90\dg-\angle C = \angle HAE$. Therefore, $\seg{TH}$ is tangent to $(AEF)$. Similarly,
\[ \angle BPT = \angle HPT = 90\dg-\angle C = \angle BAP, \]
so $\seg{TP}$ is tangent to $(ABP)$. Finally, note that 
\[ \angle MBP = \angle CBE = 90\dg-\angle C = \angle BAP, \]
so $\seg{MB}$ is tangent to $(ABP)$. Recall that $\seg{ME}$ is tangent to $(AEF)$. Then, since $TH^2=TP^2$ and $MB^2=ME^2$, both $T$ and $M$ lie on the radical axis of both circles. But $A$ lies on both circles, so this radical axis is the $A-$median, as desired. $\blacksquare$	
\end{document}
