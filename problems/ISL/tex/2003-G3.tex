\documentclass[14pt]{article}

% Packages
\usepackage{ezra}

% Title
\title{2003 G3}
\author{Ezra Guerrero Alvarez}

% Formatting
\setlength{\parindent}{0pt}

%document
\begin{document}
\maketitle
	
\section*{2003 G3}

\begin{statement}{2003 G3}
	Let $ABC$ be a triangle and let $P$ be a point in its interior.  Denote by $D$, $E$, $F$ the feet of the perpendiculars from $P$ to the lines $BC$, $CA$, $AB$, respectively.  Suppose that
	\[ AP^2 + PD^2 = BP^2 + PE^2 = CP^2 + PF^2. \]
	Denote by $I_A$, $I_B$, $I_C$ the excenters of the triangle $ABC$.  Prove that $P$ is the circumcenter of the triangle $I_AI_BI_C$.
\end{statement}
We have
\begin{align*}
AP^2 + PD^2 &= BP^2 + PE^2 \\
AP^2 - PE^2 &= BP^2 - PD^2 \\
AE^2 &= BD^2 \\
AE &= BD.
\end{align*}
Analogously, $BF = CE$ and $CD = AF$. It follows that $D,E,F$ are the extouch points. Hence, $P$ is the concurrency point of the perpendiculars to the sides through the extouch points. Now, consider a homothety with center $I$ and ratio $2$. This sends $O$ to $V$, the \emph{Bevan Point}. Since it also sends the midpoints of arcs $\widehat{BC},\widehat{CA},\widehat{AB}$ to the excenters, $V$ is the circumcenter of $\triangle I_AI_BI_C$. By the homothety, $\seg{I_AV}\perp\seg{BC}$ and analogously with the other sides, so $V$ is the concurrency point of the perpendiculars to the sides through the extouch points. Hence, $P=V$ as desired. $\blacksquare$
	
\end{document}
