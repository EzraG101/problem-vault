\documentclass[14pt]{article}

% Packages
\usepackage{ezra}

% Title
\title{2002 N3}
\author{Ezra Guerrero Alvarez}

% Formatting
\setlength{\parindent}{0pt}

%document
\begin{document}
\maketitle
	
\section*{2002 N3}

\begin{statement}{2002 N3}
	Let $p_1,p_2,\ldots,p_n$ be distinct primes greater than $3$. Show that $2^{p_1p_2\cdots p_n}+1$ has at least $4^n$ divisors.
\end{statement}
We will prove the stronger claim that $N:=2^{p_1p_2\cdots p_n}+1$ has at least $2^{2^n}$ divisors. \\
For a subset $\SA$ of $\{p_1,\ldots,p_n\}$ let $\pi(\SA)$ denote the product of its elements, with $\pi(\emptyset)=1$ by convention. Then, since $\pi(\SA)\mid p_1\cdots p_n$ and this product is odd, it is well known that
\[ 2^{\pi(\SA)}+1\mid N. \]
Now, call a prime factor $q$ of $2^{\pi(\SA)}+1$ \emph{qualified} if it does not divide $2^{\pi(\SC)}+1$ for any $\SC\subseteq\{p_1,\ldots,p_n\}$ with $\pi(\SC)<\pi(\SA)$. By Zsigmondy's, we know that there is a qualified prime for every subset of $\{p_1,\ldots,p_n\}$ (here we use $p_i>3$ to avoid the exception). Thus, since each of these qualified primes, which by definition are distinct, divide $N$, we see that $N$ has at least $2^n$ prime divisors. Therefore, $N$ has at least $2^{2^n}$ divisors. $\blacksquare$
	
\end{document}
