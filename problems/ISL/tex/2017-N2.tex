\documentclass[14pt]{article}

% Packages
\usepackage{ezra}

% Title
\title{2017 N2}
\author{Ezra Guerrero Alvarez}

% Formatting
\setlength{\parindent}{0pt}

%document
\begin{document}
\maketitle
	
\section*{2017 N2}

\begin{statement}{2017 N2}
	Let $ p \geq 2$ be a prime number. Eduardo and Fernando play the following game making moves alternately: in each move, the current player chooses an index $i$ in the set $\{0,1,2,\ldots, p-1 \}$ that was not chosen before by either of the two players and then chooses an element $a_i$ from the set $\{0,1,2,3,4,5,6,7,8,9\}$. Eduardo has the first move. The game ends after all the indices have been chosen .Then the following number is computed:
	\[ M=a_0+a_110+a_210^2+\cdots+a_{p-1}10^{p-1}= \sum_{i=0}^{p-1}a_i.10^i .\]
\end{statement}
Eduardo wins by setting $a_0=0$ always. If $p=2,5$ then this is clear, Eduardo may relax the rest of the game. Otherwise, assume $p\perp 10$ and let $d=\operatorname{ord}_p(10)$. Then, $p-1=dk$. We have two cases:
\begin{enumerate}
	\item If $d$ is even, split $[1,p-1]$ into intervals of the form $[dx+1,d(x+1)]$. If Fernando picks $a_i$, Eduardo chooses $a_j=a_i$, where $j=i\pm\frac d2$, whichever is still in the same interval as $i$. Since $10^{\frac d2}\equiv -1\pmod p$, when the game ends
	\[ M\equiv 0+\ldots+0\equiv 0\pmod p, \]
	as desired.
	\item If $d$ is odd, then $k$ must be even. Then, pair the elements of $[1,p-1]$ into $\frac k2$ pairs of numbers $d$ apart. If Fernando picks $a_i$, Eduardo chooses $a_j=9-a_i$, where $j$ is $i$'s pair. Then, $10^ia_i+10^ja_j\equiv 9\cdot 10^i$, so when the game ends
	\[ M=\ell\cdot\underbrace{99\ldots9}_{d \text{ times}}\equiv\ell\cdot(10^d-1)\equiv 0\pmod p, \]
	as desired. $\blacksquare$
\end{enumerate}
	
\end{document}
