\documentclass[14pt]{article}

% Packages
\usepackage{ezra}

% Title
\title{2018 N5}
\author{Ezra Guerrero Alvarez}

% Formatting
\setlength{\parindent}{0pt}

%document
\begin{document}
\maketitle
	
\section*{2018 N5}

\begin{statement}{2018 N5}
	Positive integers $x$, $y$, $z$, $t$ satisfy
	$xy - zt = x + y = z + t$.
	Can $xy$ and $zt$ both be perfect squares?
\end{statement}
We claim the answer is no. First, note that if $x$ and $y$ have different parity, then $z$ and $t$ have different parity as well. But then $zy-zt$ is even and $x+y$ is odd. Thus, we must have $x\equiv y\pmod2$ and $z\equiv t\pmod 2$. Thus, let
\[ p:=\frac{x+y}2,q:=\frac{x-y}2,r:=\frac{z+t}2,s:=\frac{z-t}2. \]
Then, the equation rewrites as $p=r$ and $(s-q)(s+q)=2p$. Thus, $s$ and $q$ must have the same parity as well. Letting
\[ a:=\frac{s+q}2,b:=\frac{s-q}2, \]
we see that $p=2ab$. Thus, we have
\begin{align*}
	x&=2ab+a-b\\
	y&=2ab-a+b\\
	z&=2ab+a+b\\
	t&=2ab-a-b.
\end{align*}
This characterizes all solutions to the given equation, so long as $(a,b)\ne(1,1)$ since this gives $t=0$. Now, suppose $xy=c^2$. Also, note that
\[ xy=(2ab)^2-(a-b)^2\qquad\text{ and }\qquad zt=(2ab)^2-(a+b)^2. \]
\begin{claim}
	$(c-2)^2<zt<c^2$.
\end{claim}
\begin{proof}
	The RHS is clear from the above equations. For the LHS we have that if $(a,b)\ne(1,1)$ then
	\begin{align*}
		a^2+b^2+1&<3(ab)^2\\
		(ab)^2+(a-b)^2+2ab+1&<(2ab)^2\\
		(ab+1)&<\sqrt{xy}\\
		4ab+4&<4c\\
		xy-4c+4<zt\\
		(c-2)^2<zt.
	\end{align*}
\end{proof}
Now, recall that $xy$ and $zt$ have the same parity since their difference is even. Since there are no perfect squares between $(c-2)^2$ and $c^2$ that have the same parity as $c$, this shows they cannot both be perfect squares. $\blacksquare$
	
\end{document}
