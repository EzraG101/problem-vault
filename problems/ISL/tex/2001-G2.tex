\documentclass[14pt]{article}

% Packages
\usepackage{ezra}

% Title
\title{2001 G2}
\author{Ezra Guerrero Alvarez}

% Formatting
\setlength{\parindent}{0pt}

%document
\begin{document}
\maketitle
	
\section*{2001 G2}

\begin{statement}{2001 G2}
	Consider an acute-angled triangle $ABC$. Let $P$ be the foot of the altitude of triangle $ABC$ issuing from the vertex $A$, and let $O$ be the circumcenter of triangle $ABC$. Assume that $\angle C \geq \angle B+30^{\circ}$. Prove that $\angle A+\angle COP < 90^{\circ}$.
\end{statement}
We begin by showing the following claim:
\begin{claim}
	We have
	\[ \sin\angle A\sin\angle B\cos\angle C<\frac14. \]
\end{claim}
\begin{proof}
	\begin{align*}
		\sin\angle A\sin\angle B\cos\angle C&=\frac12\sin\angle A(\sin(\angle B+\angle C)+\sin(\angle B-\angle C))\\
		&=\frac12\sin\angle A(\sin\angle A-\sin(\angle B-\angle C))\\
		&\le\frac12\sin\angle A\left(\sin\angle A-\frac12\right)\\
		&<\frac14,
	\end{align*}
where the last inequality comes from the quadratic on $\sin\angle A$ being increasing on $[1/2,1)$ and negative when $\sin\angle A<\frac12$.
\end{proof}
Now, this claim implies
\[ r^2=\frac{ab}{4\sin\angle A\sin\angle B}>ab\cos\angle C=a\cdot PC. \]
Now, from power of a point
\[ OP^2=r^2-PC\cdot PB=r^2-a\cdot PC+PC^2>PC^2, \]
so $OP>PC$. Thus, looking at triangle $POC$, we get $90\dg-\angle A=\angle PCO>\angle COP$ as desired. $\blacksquare$
	
\end{document}
