\documentclass[14pt]{article}

% Packages
\usepackage{ezra}

% Title
\title{2015 A2}
\author{Ezra Guerrero Alvarez}

% Formatting
\setlength{\parindent}{0pt}

%document
\begin{document}
\maketitle
	
\section*{2015 A2}

\begin{statement}{2015 A2}
	Determine all functions $f:\mathbb{Z}\rightarrow\mathbb{Z}$ with the property that
	\[ f(x-f(y))=f(f(x))-f(y)-1 \]
	holds for all $x,y\in\mathbb{Z}$.
\end{statement}
We claim the only solutions are $f\equiv-1$ and $f\equiv x+1$ which are easily seen to work. Now, plugging in $x=0,y=f(0)$ we obtain
\[ f(-f(f(0)))=-1. \]
Let $u=-f(f(0))$, then setting $y=u$,
\[ f(x+1)=f(f(x)). \]
Hence, the given equation becomes $f(x-f(y))=f(x+1)-f(y)-1$. Substituting $x=f(n)-1,y=n$ we get
\begin{align*}
	f(f(n)-1-f(n))&=f(f(n)-1+1)-f(n)-1\\
	f(-1)&=f(n+1)-f(n)-1\\
	f(-1)+1&=f(n+1)-f(n).
\end{align*}
Now, since the LHS is a constant, we get $f(n+1)-f(n)$ is constant for all $n$. Since the domain of $f$ is $\mathbb Z$, it follows $f(x)=kx+c$ for some $k,c\in\mathbb Z$. Substituting in the equation we find
\[ c+1=(k^2-k)(x+y)+2kc. \]
Setting $x+y=0$, we find $c\mid c+1$, so $c\in\{-1,1\}$. Each of these cases give the aforementioned solutions. $\blacksquare$
	
\end{document}
